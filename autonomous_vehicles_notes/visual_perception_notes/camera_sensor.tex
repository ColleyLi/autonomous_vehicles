\section{Camera Sensor}
\label{camera_sensor}

The camera sensor is one of the primary sensors in a vehicle's sensor suite.
This is because the camera is a rich sensor that captures incredible detail
about the environment around the vehicle. However, it requires
extensive processing to make use of the information that is
available in that image. 

In this section, we will highlight why the camera is a critical sensor
for autonomous driving. We will then briefly introduce the concept of
image formation and present the pinhole camera model
which captures the essential elements of how a camera works in a simple
and elegant manner.

Of all the common self-driving car sensors, the camera is the sensor
that provides the most detailed appearance information from objects in the environment. Appearance information
is particularly useful for scene understanding tasks such as object detection, segmentation and identification. 
Appearance information is what allows us to distinguish between road signs or
traffic lights states, to track turn signals and resolve overlapping vehicles
into separate instances. Because of its high resolution output, the camera is able to collect and provide orders of magnitude, more information than
other sensors used in self-driving while still being relatively inexpensive. The combination of high valued
appearance information and low cost make the camera an essential component of our sensor suite. 


\subsection{The Pinhole Camera Model}
\label{pinhole_model}

Let us see how the camera manages to collect this huge amount
of information. A camera is a passive
external receptive sensor. It uses an imaging sensor
to capture information conveyed by light rays emitted
from objects in the world. This was originally done with
film but nowadays we use rather sophisticated
silicon chips to gather this information. Light is reflected
from every point on an object in all directions, and a portion of these rays travel towards the camera sensor. 

Look at the car's reflected rays collected by our imaging surface. Do you think we will get
a good representation of the car on the image sensor
from this ray-pattern? Unfortunately, no. Using this basic open
sensor camera design, we will end up with blurry images because our imaging
sensor is collecting light rays from multiple points on the object at the same
location on the sensor. The solution to our problem is to put a barrier in front of the imaging sensor
with a tiny hole or aperture in its center. The barrier allows only a small number of light rays to pass
through the aperture, reducing the blurriness of the image. This model is called the \textbf{pinhole camera model} and describes the relationship between a point in
the world and it's corresponding projection on the image plane, see for instance \cite{PinholeCameraWiki}. 

The two most important parameters in a pinhole camera model are the distance between
the pinhole and the image plane which we call the focal length and is typically denoted with $f$. 
The focal length defines the size of the object projected
onto the image and plays an important role
in the camera focus when using lenses to improve camera performance. 

\begin{framed}
\begin{remark}{\textbf{Focal Lenght $f$}}

Specifically, we define the focal length $f$
as the distance between the camera and the
image coordinate frames along the $z$-axis of
the camera coordinate frame.
\end{remark}
\end{framed}


The coordinates of
the center of the pinhole, $(c_u, c_v)$, which we call the camera center, these coordinates to
find the location on the imaging sensor that the object projection will inhabit. 

Although the pinhole camera model is very simple, it works surprisingly well for representing the image
creation process. By identifying
the focal length and the camera's center for
a specific camera configuration, we can mathematically describe
the location that a ray of light emanating from an object in the world will strike
the image plane. This allows us to form a measurement model of image formation for use in state estimation
and object detection. 


\begin{framed}
\begin{remark}{\textbf{Some History}}

A historical example of the pinhole camera model
is the camera obscura, which translates to
dark room camera in English. Historical evidence shows that this form of imaging
was discovered as early as 470 BC in ancient
China and Greece. It's simple construction with a pinhole aperture in front of an imaging surface makes it
easy to recreate on your own, and is in fact a safe way to watch solar eclipse if
you're so inclined.
\end{remark}
\end{framed}


Nowadays  cameras allow us to collect extremely high resolution data. They can operate in
low-light conditions or at a long range due to the advanced lens optics
that gather a large amount of light and focus it
accurately on the image plane. The resolution and sensitivity of camera sensors
continues to improve, making cameras one of the most ubiquitous sensors on the planet; think for example how many cameras you me be owning.

These advances are also
extremely beneficial for understanding the environment around a self-driving car.  
Cameras specifically designed for autonomous vehicles need to work well in a wide range of lighting conditions and
in distances to objects. These properties are essential to driving safely in all operating conditions. 


\subsection{Summary}

This was an introductory section. We discussed the usefulness of the camera as a sensor
for autonomous driving. We also saw the pinhole
camera model in its most basic form, which we'll use
to construct algorithms for visual perception. In the next section, we will describe how an image is formed, a process referred to
as projective geometry, which relates objects in the world to their projections on the imaging sensor.

\subsection{Questions}

\begin{enumerate}
\item Describe the pinhole camera model. Why is it useful?
\end{enumerate}

\subsection{Assignements}

\begin{enumerate}
\item Using OpenCV read and display an image.
\end{enumerate}

