\section{Sensor Calibration}
\label{sensor_calibration}

Now that we've seen how we can combine multiple sources
of sensor data to estimate the vehicle state, it's time to address the topic of \textbf{sensor calibration}.
Sensor calibration and is absolutely essential for doing state estimation properly. Concretely, in this section,
we will discuss the three main types of sensor calibration and why we need to think about
them when  designing a state estimator for a self-driving car. The three main types
of calibration will talk about are

\begin{itemize}
\item Intrinsic calibration, which deals with sensors specific parameters
\item Extrinsic calibration, which deals with how the sensors are positioned and oriented
on the vehicle
\item Temporal calibration, which deals with the time offset between different sensor
measurements. 
\end{itemize}

\subsection{Intrinsic calibration}
\label{intrinsic_calibration}
Let's look at intrinsic calibration first. In intrinsic calibration, we want to determine the fixed parameters
of our sensor models, so that we can use them in an estimator like an extended Kalman filter. Every sensor
has parameters associated with it that are unique to that
specific sensor and are typically expected to be constant. 

For example, we might have an encoder attached to one axle of the car that measures the wheel rotation
rate $\omega$. If we want to use $\omega$ to estimate the forward velocity
$v$ of the wheel, we would need to know the radius $R$ of the wheel, so that we can use
the following equation 

\begin{equation}
v = \omega R
\end{equation}

In this case, $R$ is a parameter of the sensor model that is specific to the wheel the encoder is attached to
and we might have a different $R$ for
a different wheel. Another example of an intrinsic sensor
parameter is the elevation angle of a scan line in a LiDAR sensor
like the Velodyne. The elevation angle is a fixed quantity but we
need to know it ahead of time so that we can properly interpret
each scan. 

So, how do we determine intrinsic parameters like these? Well, there are
a few practical strategies for doing this. The easiest one
is just let the manufacturer do it for you. Often, sensors
are calibrated in the factory and come with a spec sheet that tells you
all the numbers you need to plug into your model to make sense of the measurements. This is usually
a good starting point but it will not always
be good enough to do really accurate state estimation because no two sensors are exactly alike
and there will be some variation in the true values of the parameters. Another easy strategy
that involves a little more work is to try measuring these
parameters by hand. This is pretty straightforward for something like a tire, but not so straightforward for something like a LiDAR where it's not exactly
practical to poke around with a protractor
inside the sensor. 

A more sophisticated approach involves estimating the intrinsic
parameters as part of the vehicle state, either on the fly or more commonly as a special calibration step before putting the
sensors into operation. This approach has the advantage of producing an accurate
calibration that's specific to the particular sensor and can also be
formulated in a way that can handle the parameters varying slowly over time. 

For example, if you continually estimate the radius
of your tires, this could be a good way of detecting when
you have a flat. Now, because the estimators
we've talked about in this course are general purpose, we already have the tools to do this kind of automatic calibration. 
In order to see how this works, let's come back to our example of a car moving in one dimension. 

we have attached an encoder to the back wheel to measure the wheel
rotation rate. If we want to estimate the wheel radius along with position
and velocity, all we need to do is add it to the state vector and work out what
the new motion and observation model should be. For the motion model, everything is the same as before except now there's
an extra row and column in the matrix
that says that the wheel radius
should stay constant from one time step to the next. For the observation model, we are still
observing position directly through GPS but now we're
also observing the wheel rotation rate through the encoder. So, we include the extra non-linear observation
in the model. From here, we can use the extended or
unscented Kalman filter to estimate
the wheel radius along with the position and velocity of the vehicle. 

Thus, intrinsic calibration is essential for doing state estimation with
even a single sensor. 

\subsection{Extrinsic calibration}
\label{extrinsic_calibration}

Extrinsic calibration is equally important for fusing information
from multiple sensors. In extrinsic calibration, we are interested in determining the relative
poses of all of the sensors usually with respect to
the vehicle frame. 

For example, we need to know the relative pose of
the IMU and the LiDAR. The rates reported by the IMU are expressed in the same coordinate system as the LiDAR point clouds. Just like with
intrinsic calibration, there are different
techniques for doing extrinsic
calibration. If you are lucky, you
might have access to an accurate CAD model of the vehicle, where all of
the sensor frames have been nicely
laid out for you. If you are less
lucky, you might be tempted to try
measuring by hand. Unfortunately, this is often difficult or
impossible to do accurately since
many sensors have the origin of
their coordinate system inside the sensor itself, and you probably do not
want to dismantle your car and all
of the sensors. 

Fortunately, we can use a similar trick
to estimate the extrinsic
parameters by including them
in our state. This can become a
bit complicated for arbitrary sensor
configurations, and there is still a lot of research
being done into different techniques for doing this reliably. 


\subsection{Temporal calibration}

Finally, an often overlooked but still important type of calibration is
temporal calibration. In all of
our discussion of multisensory fusion, we've been
implicitly assuming that all of the
measurements we have combined are captured exactly the same moment in time or at least close enough for a given
level of accuracy. But how do we decide whether
two measurements are close enough to be considered synchronized? Well, the obvious
thing to do would just be to timestamp each measurement
when the on-board computer receives it, and match up the
measurements that are closest
to each other. 

For example, if we get LiDAR scans at 15 hertz and IMU
readings at 200 hertz, we might want to pair
each LiDAR scan with the IMU reading
whose timestamp is the closest match. 


In reality, there is an unknown delay
between when the LiDAR or IMU actually records an observation and when it arrives at
the computer. These delays can be caused by the time it takes for the sensor data to be transmitted to
the host computer, or by pre-processing steps performed by
the sensor circuitry, and the delay
can be different for different sensors. Hence, if we want to get a really accurate
state estimate, we need to think about how well our sensors are
actually synchronized, and there are different ways to approach this. The simplest and most
common thing to do is just to assume
the delay is zero. You can still get a working estimator
this way, but the results may be less accurate than
what you would get with a better temporal calibration. 

Another common strategy is to use hardware timing signals to synchronize
the sensors, but this is often
an option only for more expensive sensor setups. As you may have guessed, it's also possible
to try estimating these time delays as part of the vehicle state, but this can get
complicated.


To summarize,
sensor fusion is impossible
without calibration. In this section, we touched upon three types of calibration.
Intrinsic calibration, which deals with
calibrating the parameters of
our sensor models. Extrinsic calibration,
which gives us the coordinate
transformations we need to transform
sensor measurements into a common
reference frame. Temporal calibration, which deals with
synchronizing measurements to
ensure they all correspond to
the same vehicle state. While there are some standard
techniques for solving all of
these problems, calibration is still very much an active area
of research. 

\section{Questions}

\begin{enumerate}
\item What is intrinsic calibration?
\item Why do we need extrinsic calibration?
\item What is temporal calibration?
\end{enumerate}
