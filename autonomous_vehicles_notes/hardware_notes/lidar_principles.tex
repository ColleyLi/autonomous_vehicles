\section{LiDAR Principles}
\label{lidar_principles}
Welcome to module four of the course. In this module, we'll
be talking about LIDAR, or light detection and ranging sensors. LIDAR has been an enabling technology
for self-driving cars because it can see in all directions and is able to provide very accurate range information. In fact, with few exceptions, most self-driving cars
on the road today are equipped with some type of LIDAR sensor. In this module, you'll learn about the operating principles
of LIDAR sensors, basic sensor models used to work with
LIDAR data and LIDAR point clouds, different kinds of transformation
operations applied to point clouds, and how we can use LIDAR to localize a self-driving car using a technique
called point cloud registration. In this video specifically, we'll explore how a LIDAR works and take a look at sensor models
for 2D and 3D LIDARs. We'll also describe the sources of measurement noise and
errors for these sensors. In future lessons, we'll discuss
in more detail how we can use LIDAR data for state estimation
of self-driving vehicles. If you've ever seen
a self-driving car like the Waymo vehicle or an Uber car, you've probably noticed something
spinning on the roof of the car. That something is a LIDAR, or light detection and ranging sensor, and its job is to provide detailed 3D scans of
the environment around the vehicle. In fact, LIDAR is one of
the most common sensors used on self-driving cars and
many other kinds of mobile robots. LIDARs come in many
different shapes and sizes, and can measure the
distances to a single point, a 2D slice of the world, or perform a full 3D scan. Some of the most popular
models used today are manufactured by firms such
as Velodyne in California, Hokuyo in Japan, and SICK in Germany. In this video, we'll mainly focus on the Velodyne sensors as
our example of choice, but the basic techniques applied
to other types of LIDARs as well. Before we get into
the nitty-gritty of LIDAR sensing, let's take a quick look at
the history of this important sensor. LIDAR was first introduced in the 1960s, not long after the invention
of the laser itself. The first group to use
LIDAR were meteorologists at the US National Center
for Atmospheric Research, who deployed LIDAR to measure
the height of cloud ceilings. These ground-based
celiometers are still in use today not only to
measure water clouds, but also to detect volcanic ash
and air pollution. Airborne LIDAR sensors are
commonly used today to survey and map the earth's
surface for agriculture, geology, military, and other uses. But the application that
first brought LIDAR into the public
consciousness was Apollo 15, the fourth manned mission
to land on the moon, and the first to use a laser altimeter
to map the surface of the moon. So, we've seen that LIDAR can be used to measure distances and create
a certain type of map, but how did they actually work, and how can we use them
onboard a self-driving car? To build a basic LIDAR in one dimension, you need three components; a laser, a photodetector, and a very precise stopwatch. The laser first emits
a short pulse of light usually in the near infrared frequency band
along some known ray direction. At the same time,
the stopwatch begins counting. The laser pulse travels
outwards from the sensor at the speed of light and
hits a distant target. Maybe another vehicle in
front of us on the road or a stationary object like
a stop sign or a building. As long as the surface of the target
isn't too polished or shiny, the laser pulse will scatter off
the surface in all directions, and some of that reflected light will travel back along
the original ray direction. The photodetector catches that return
pulse and the stopwatch tells you how much time has passed between when the pulse first went out
and when it came back. That time is called the round-trip time. Now, we know the speed of light, which is a bit less than 300
million meters per second. So, we can multiply the speed of
light by the round-trip time to determine the total round trip distance
traveled by the laser pulse. Since light travels
much faster than cars, it's a good approximation to think
of the LIDAR and the target as being effectively stationary
during the few nanoseconds that it takes for all of this to happen. That means that the distance from
the LIDAR to the target is simply half of the round-trip distance
we just calculated. This technique is called
time-of-flight ranging. Although it's not the only way
to build a LIDAR, it's a very common method
that also gets used with other types of ranging sensors
like radar and sonar. It's worth mentioning that the photodetector also
tells you the intensity of the return pulse relative to the
intensity of the pulse that was emitted. This intensity information is less
commonly used for self-driving, but it provides some extra information
about the geometry of the environment and the material
the beam is reflecting off of. So, why is intensity data useful? Well in part, it turns out that it's
possible to create 2D images from LIDAR intensity data that
you can then use the same computer vision algorithms you'll
learn about in the next course. Since LIDAR is its own light source, it actually provides a way for
self-driving cars to see in the dark. So, now we know how to
measure a single distance to a single point using
a laser, a photodetector, a stopwatch, and
the time-of-flight equation, but obviously it's not enough to stay laser focused on a single point ahead. So, how do we use
this technique to measure a whole bunch of
distances in 2D or in 3D? The trick is to build
a rotating mirror into the LIDAR that directs the emitted pulses
along different directions. As the mirror rotates, you can measure distances at points
in a 2D slice around the sensor. If you then add an up
and down nodding motion to the mirror along with the rotation, you can use the same principle
to create a scan in 3D. For Velodyne type LIDARs, where the mirror rotates
along the entire sensor body, it's much harder to use
a knotting motion to make a 3D scan. Instead, these sensors
will actually create multiple 2D scan lines from a series of individual lasers spaced at
fixed angular intervals, which effectively lets you paint the world with horizontal
stripes of laser light. Here's an example of
a typical raw LIDAR stream from a Velodyne sensor
attached to the roof of a car. The black hole in the middle is a blind spot where the sensor
itself is located, and the concentric circles
spreading outward from there are the individual scan lines produced
by the rotating Velodyne sensor. Each point in the scan is colored by
the intensity of the return signal. The entire collection of points in
the 3D scan is called a point cloud, and we'll be talking about
how to use point clouds for state estimation in the next few videos. But before we get to point clouds, we need to think about
individual points in 3D. Now typically, LIDARs
measure the position of points in 3D using
spherical coordinates, range or radial distance from
the center origin to the 3D point, elevation angle measured up
from the sensors XY plane, and azimuth angle, measured
counterclockwise from the sensors x-axis. This makes sense because
the azimuth and elevation angles tell you the direction
of the laser pulse, and the range tells you how far in that direction
the target point is located. The azimuth and elevation angles
are measured using encoders that tell you
the orientation of the mirror, and the range is measured using the time
of flight as we've seen before. For Velodyne type LIDARs, the elevation angle is fixed
for a given scan line. Now, suppose we want to
determine the cartesian XYZ coordinates of our scanned point
in the sensor frame, which is something we often
want to do when we're combining multiple
LIDAR scans into a map. To convert from spherical
to Cartesian coordinates, we use the same formulas you would've encountered in your mechanics classes. This gives us an inverse sensor model. We say this is the inverse model because our actual measurements are
given in spherical coordinates, and we're trying to
reconstruct the Cartesian coordinates at the points
that gave rise to them. Note that we haven't said anything
about measurement noise yet, we'll come back to that in just a moment. To go the other way from Cartesian
coordinates to spherical coordinates, we can work out the inverse
transformation given here. This is our forward sensor
model for a 3D LIDAR, which given a set of Cartesian coordinates defines what
the sensor would actually report. Now, most of the time the
self-driving cars we're working with use 3D LIDAR sensors
like the Velodyne, but sometimes you might want
to use a 2D LIDAR on its own, whether for detecting obstacles
or for state estimation in more structured environments
such as parking garages. Some cars have multiple 2D LIDARs strategically placed to
act as a single 3D LIDAR, covering different areas with a greater
or lesser density of measurements. For 2D LIDARs, we use exactly the same
forward and inverse sensor models. But the elevation angle
enhance the z component of the 3D point in the sensor
frame are both zero. In other words, all of
our measurements are confined to the XY plane of the sensor, and our spherical
coordinates collapsed to the familiar 2D polar coordinates. We've seen now how to convert between
spherical coordinates measured by the sensor and the cartesian coordinates that we'll typically be
interested in for state estimation, but what about measurement noise? For LIDAR sensors, there are several important sources
of noise to consider. First, there is uncertainty in the exact time of arrival
of the reflected signal, which comes from the fact that
the stopwatch we use to compute the time of flight necessarily
has a limited resolution. Similarly, there is uncertainty in the exact orientation
of the mirror in 2D and 3D LIDARs since the encoder is used to measure
this also have limited resolution. Another important factor
is the interaction with the target surface which can
degrade the return signal. For example, if the surface
is completely black, it might absorb most of the laser pulse. Or if it's very shiny like a mirror, the laser pulse might be scattered completely away from
the original pulse direction. In both of these cases, the LIDAR will typically
report a maximum range error, which could mean there is empty space
along the beam direction, or that the pulse encountered a highly absorptive or
highly reflective surface. In other words, you simply can't
tell if something is there or not, and this can be a problem for
safety if your self-driving car is relying on LIDAR alone to
detect and avoid obstacles. Finally, the speed of light actually varies depending on the material
it's traveling through. The temperature and humidity
of the air can also suddenly affect the speed of light in our
time-of-flight calculations for example. These factors are
commonly accounted for by assuming additive zero-mean
Gaussian noise on the spherical coordinates with an empirically determined or
manually tuned covariance. As we've seen before, the Gaussian noise model is
particularly convenient for state estimation even if it's not
perfectly accurate in most cases. Another very important source
of error that can't be accounted for so easily
is motion distortion, which arises because the vehicle
the LIDAR is attached to is usually moving relative to
the environment it's scanning. Now, although the car is unlikely to be moving at an appreciable fraction
of the speed of light, it is often going to be moving at an appreciable fraction of
the rotation speed of the sensor itself, which is typically around 5-20 hertz when scanning objects at distances
of 10 to a 100 meters. This means that every single point
in a LIDAR sweep is taken from a slightly different position and
a slightly different orientation, and this can cause artifacts such as duplicate objects to
appear in the LIDAR scans. This makes it much harder for a self-driving car to
understand its environment, and correcting this motion
distortion usually requires an accurate motion model for the vehicle provided by GPS and INS for example. To recap, LIDAR sensors
measure distances by emitting pulse laser light and measuring
the time of flight of the pulse. 2D or 3D LIDAR is extend
this principle by using a mirror to sweep the laser across the environment and measure distances in many directions. In the next video, we'll look more closely at the point
clouds created by 2D and 3D LIDARs, and how we can use them for state
estimation on-board our self-driving car.
