\section{Feed Forward Neural Networks (FFNN) }
\label{feed_forward_nn}


\subsection{Definitions}

The goal of a FFNN is to approximate some function $f^{*}$. For example,
a classifier, $y = f^{*}(\mathbf{x})$ maps an input $\mathbf{x}$ to a category $y$.
Thus, a FFNN defines a mapping

\begin{equation}
\mathbf{y} = f(\mathbf{x};\mathbf{\theta})
\end{equation}
Its objective is to learn the parameters $\mathbf{\theta}$ that result in the best function approximation.


\begin{framed}
\begin{remark}

We call these model feedforward because information flows thriugh the function being
evaluated from $\mathbf{x}$ through the intermediate computations used to define $f$ and finally to
output $\mathbf{y}$. There ae no feedback connections in which outputs of the model are fed back into itself.
When feedforward neural networks are extended to include feedback connections, they are called recurrent neural networks
\end{remark}
\end{framed}

Feedforward networks are of extreme importance to machine learning practi-
tioners. They form the basis of many important commercial applications. For
example, the convolutional networks used for object recognition from photos are a
specialized kind of feedforward network. Feedforward networks are a conceptual
stepping stone on the path to recurrent networks, which power many natural
language applications.


FFNNs are called networks because they are typically represented by composing together
many different functions. The model is associated with a directed acyclig graph describing
how the functions are composed together.  For example, we might have four functions $f^{(i)}, i=1,\ldots,4$
connected in a chain to form 

\begin{equation}
f(\mathbf{x}) = f^{(4)}(f^{(3)}(f^{(2)}(f^{(1)}(\mathbf{x}))))
\end{equation}

These chain structures are the most commonly used structures of neural networks. The overal length of the chain gives
the depth of the model. The final layer is called the output layer. 

During training we drive $f(\mathbf{x})$ to match $f^{*}(\mathbf{x})$. We do so by using a training dataset which contains noisy examples of 
$f^{*}(\mathbf{x})$ evaluated at different points. As usual with supervised learning, each example is accompanied with a label such that

\begin{equation}
y \approx f^{*}(\mathbf{x}) 
\end{equation}

Although the training examples specify in some sense what the output layer should do, i.e. produce a value that is close to $y$, they do not
specify directly the behavior of the other layers. The learning algorithm must decide
how to use those layers to produce the desired output, but the training data do
not say what each individual layer should do. It is the job of the learning algorithm to decide how these so called \textbf{hidden layers} best implement
an approximation of $f^{*}$. 

Each hidden layer of the network is typically vector valued. The
dimensionality of these hidden layers determines the
width of the model. Each element of the vector may be interpreted as playing a role analogous to a neuron

