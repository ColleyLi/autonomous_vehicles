\section{Feed Forward Neural Nets for Autononmous Vehicles}
\label{feed_forward_neural_nets_autononmous_vehicles}

This section discusses a topic that has changed the way we think about autonomous perception, namely
artificial neural networks (ANNs). Throughout this module, you will learn
how these algorithms can be used to build a self-driving car perception stack, and you'll learn the different components to
design and train a deep neural network. Now we won't be able to teach you
everything you need to know about artificial neural networks, but this module is a good
introduction to the field. 


In this lesson, you will learn about the building blocks of feedforward neural networks, FFNN, a very useful
basic type of artificial neural network. Specifically, we'll look at the hidden
layers of a feedforward neural network. The hidden layers are important, as they
differentiate the mode of action of neural networks from the rest of
machine learning algorithms. We'll begin by looking at the mathematical
definition of feedforward neural networks, so you can start to understand
how to build these algorithms for the perception stack. 

A feedforward neural network defines a mapping from an input $\mathbf{x}$ to an output $\mathbf{y}$ through
a function $f(\mathbf{x}; \boldsymbol{theta})$. For example, we use neural networks
to produce outputs such as the location of all cars in a camera image. The function $f$ takes an input $\mathbf{x}$, and
uses a set of learned parameters $\boldsymbol{theta}$, to interact with $\mathbf{x}$,
resulting in the output $\mathbf{y}$. The concept of learned parameters is
important here, as we do not start with the correct form of the function $f$, which
maps our inputs to our outputs directly. Instead, we must construct an
approximation to the true function using a generic neural network. This means that neural networks can be
thought of as function approximators. Usually we describe a feedforward neural
network as a function composition. 

In a sense,
each function $f_i$ is a layer on top of the previous function, $f_{i- 1}$. Usually we have $N$ functions in our
compositions where $N$ is a large number, stacking layer upon layer for
improved representation. This layering led to
the name deep learning for the field describing these
sequences of functions. Now let us describe this
function composition visually. 


Here you can see a four-layer feedforward neural network. This neural network has an input layer
which describes the data input x to the function approximator. Here x can be a scalar,
a vector, a matrix or even a n-dimensional tensor such as images. The input gets processed by the first
layer of the neural network, the function $f_1( \mathbf{x})$. We call this layer the first hidden layer. Similarly, the second hidden layer
processes the output of the first hidden layer through the function $f_1( \mathbf{x})$. 
We can add as many hidden layers as we'd like, but each layer adds additional parameters to be learned, and more
computations to be performed at run time. We will discuss how the number of hidden
layers affects the performance of our system later on in the course. The final layer of the neural
network is called the output layer. It takes the output of
the last hidden layer and transforms it to a desired output $\mathbf{y}$. 

Now, we should have the intuition on why these networks are called feedforward. This is because information flows from the
input $\mathbf{x}$ through some intermediate steps, all the way to the output $\mathbf{y}$
without any feedback connections.

Now let us go back to the network definition and check out how our visual representation
matches our function composition. In this expression we see $\mathbf{y}$,
which is called the input layer. We see the outer most function $f_{N}$,
which is the output layer. And we see each of the functions
$f_1$ to $f_{N-1}$ in between, which are the hidden layers. Now before we delve deeper into these
fascinating function approximators, let's look at a few examples of how we
can use them for autonomous driving. Remember, this course is
on visual perception, so we'll restrict our input
$\mathbf{x}$ to always be an image. The most basic perception task
is that of classification. Here we require the neural network to
tell us what is in the image via a label. We can make this task more complicated
by trying to estimate a location as well as a label for objects in the scene. This is called object detection. Another set of tasks we might be
interested in are pixel-wise tasks. As an example we might want to estimate a
depth value for every pixel in the image. This will help us determine
where objects are. Or, we might want to determine
which class each pixel belongs to. This task is called semantic segmentation,
and we'll discuss this in depth along with
object detection later in the course. In each case, we can use a neural network
to learn the complex mapping between the raw pixel values from the image
to the perception outputs we're trying to generate, without having to
explicitly model how that mapping works. This flexibility to represent
hard-to-model processes is what makes neural networks so popular. Now let's take a look at how to learn
the parameters needed to create robust perception models. During a process referred to
as Neural Network Training, we drive the neural network function f of
(x) and theta to match a true function f*(x) by modifying the parameters
theta that describe the network. The modification of theta is done by
providing the network pairs of input x and its corresponding true out output f*(x). We can then compare the true output to
the output produced by the network and optimize the network parameters
to reduce the output error. Since only the output of the neural
network is specified for each example, the training data does not specify what
the network should do with its hidden layers. The network itself must decide how
to modify these layers to best implement an approximation of f*(x). As a matter of fact, hidden units are what
make neural networks unique when compared to other machine learning models. So let us define more clearly
the hidden layer structure. The hidden layer is comprised of
an affine transformation followed by an element wise non-linear function g. This non-linear function is
called the activation function. The input to the nth
hidden layer is h of n- 1, the output from the previous hidden layer. In the case where the layer
is the first hidden layer, its input is simply the input image x. The affine transformation is
comprised of a multiplicative weight matrix W, and
an additive bias Matrix B. These weights and biases are the learn parameters theta in
the definition of the neural network. Finally, the transformed input is passed
through the activation function g. Most of the time, g does not contain
parameters to be learned by the network. As an example, let us take a look at
the rectified linear unit, or ReLU, the default choice of activation functions
in most neural networks nowadays. ReLUs use the maximum between zero and the output of the affine transformation as
their element-wise non-linear function. Since they are very similar to linear
units, they're quite easy to optimize. Let us go through an example of
a ReLU hidden-layer computation. We are given the output of
the previous hidden layer hn- 1, the weight matrix W,
and the bias matrix b. We first need to evaluate
the affine transformation. Remember, the weight matrix is
transposed in the computation. Let's take a look at the dimensions of
each of the matrices in this expression. hn- 1 is a 2x3 matrix in this case. W is a 2x5 matrix. The final result of our affine
transformation is a 5 by 3 matrix. Now, let us pass this matrix
through the ReLU non-lineary. We can see that the ReLU prevents
any negative outputs from the affine transformation from
passing through to the next layer. There are many additional activation
functions that can be used as element wise non-linearities in hidden layers for
neural networks. In fact, the design of hidden units is
another extremely active area of research in the field and does not yet
have many guiding theoretical principles. As an example, certain neural network architectures
use the sigmoid non-linearity, the hyperbolic tangent non-linearity,
and the generalization of ReLU, the maxout non-linearity as their
hidden layer activation functions. If you're interested in learning more
about neural network architectures, I strongly encourage you to check out
some of the deep learning courses offered on Coursera. They're amazing. In this lesson, you learned the main building blocks
of feedfoward neural networks including the hidden layers that comprise the core
of the machine learning models we use. You also learned different types of
activation functions with ReLUs being the default choice for
many practitioners in the field. In the next lesson,
we'll explore the output layers and then study how to learn the weights and
bias matrices from training data, setting the stage for training our first
neural network later on in the module. [MUSIC]