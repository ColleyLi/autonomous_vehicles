\section{Hardware Configuration}
\label{hardware_configuration}
Section \ref{hardware_for_self_driving_cars} Hardware for Self-driving Cars, covers the various kinds of sensors most commonly used for perception. 
One question that should be answered is how do we place these sensors on the vehicle in order to 
aquire a a complete view of the environment? 
In this section, we will discuss the configuration design to meet sensor coverage needs for an autonomous driving car. 
We will do this by going through two common scenarios, driving on a highway and driving in an urban environment. 
After analyzing these scenarios, we will lay out the overall coverage requirements and discuss some issues with the design. 

Let's however begin by recalling the most commonly available sensors from our last video. 
These are:

\begin{itemize}
\item  The camera for appearance input. 
\item The stereo camera for depth information
\item Lidar for all whether 3D input
\item Radar for object detection
\item Ultrasonic for short-range 3D input 
\item GNSS/IMU data and wheel odometry for ego state estimation. 
\end{itemize}

Also, remember that all of these sensors come in different configurations and  different ranges in fields of view over which they can sense. 
They have some resolution that depends on the instrument specifics and field of view. 
Before we move to discussing coverage, let's define the deceleration rates we're willing to accept for driving which will drive the detection ranges needed for our sensors. 

{textbf{Aggressive Deceleration}}


\begin{framed}
\theoremstyle{remark}
\begin{remark}{\textbf{Aggressive Deceleration}}

Aggressive deceleration are set at five meters per second squared which is roughly the deceleration you experience when you slam the brakes hard and try to stop abruptly in case of an emergency. 
\end{remark}
\end{framed}

Normal decelerations are set to two meters per second squared, which is reasonably comfortable while still allowing the car to come to a stop quickly. 
Given a constant deceleration our braking distance $d$ can be computed as follows. $D$ is equal to $V^2/2a$, where $V$ is the vehicle velocity and 
$a$ is its rate of deceleration. We can also factor in reaction time of the system and road surface friction limits, but we'll keep things simple in this discussion. L
et's talk about coverage now. The question we want to answer is where should we place our sensors so that we have sufficient input for our driving task? 
Practically speaking, we want our sensors to capture the ODD we have in mind or the ODD our system can produce decisions for. 
We should be able to provide all of the decisions with sufficient input. 
There can be so many possible scenarios in driving but we'll look at just two common scenarios to see how the requirements drive our sensor selection. 
Will look at highway and urban driving. Let's think about these two situations briefly. 


For a divided highway, we have fast moving traffic, usually high volume, and quite a few lanes to monitor, 
but all vehicles are moving in the same direction. The other highlight of driving on a highway setting is that there are 
fewer and gradual curves and we have exits and merges to consider as well. 

On the other hand, in the urban situation we'll consider, 
we have moderate volume and moderate speed traffic with fewer lanes but with 
traffic moving in all directions especially through intersections. 

\subsubsection{Highway Scenario}

Let's start with the highway setting. 
We can break down the highway setting into three basic maneuver needs. 

\begin{itemize}
\item We may need to hit the brakes hard if there's an emergency situation. 
\item We need to maintain a steady speed matching the flow of traffic around us.
\item We might need to change lanes. 
\end{itemize}

In the case of an emergency stop, if there is a blockage on our road we want to stop in time. 
So, applying our stopping distance equation longitudinally, we need to be able to sense about a 110 meters in front of us assuming a highway speed of a 120 kilometers and aggressive deceleration. 
Most self-driving systems aim for sensing ranges of a 150 to 200 meters in front of the vehicle as a result. 
Similarly, to avoid lateral collision or to change lanes to avoid hitting an obstacle in our lane, we need to be able to sense at least our adjacent lanes, 
which are 3.7 meters wide in North America. To maintain speed during vehicle following, we need to sense the vehicle in our own lane. 
Both their relative position and the speed are important to maintain a safe following distance. 
This is usually defined in units of time for human drivers and set to two seconds in nominal conditions. 
It can also be assessed using aggressive deceleration of the lead vehicle and the reaction time from our ego vehicle. 
So, at a 120 kilometers per hour, relative position and speed measurements to a range of 165 meters are needed and typical systems use 100 meters for this requirement. 
Laterally, we need to know what's happening anywhere in our adjacent lanes in case another vehicles seeks to merge into our lane or we need to merge with other traffic. 
A wide 160 to 180 degree field of view is required to track adjacent lanes and a range of 40 to 60 meters is needed to find space between vehicles. 

Finally, let's discuss the lane change maneuver and consider the following scenario. 
Suppose we want to move to the adjacent lane, longitudinally we need to look forward, so we are a safe distance from the leading vehicle and 
we also need to look behind just to see what the rear vehicles are doing and laterally it's a bit more complicated. 
We may need to look beyond just the adjacent lanes. For example, what if a vehicle attempts to maneuver into the adjacent lane at the same time as we do? 
We'll need to coordinate our lane change room maneuvers so we don't crash. 

The sensor requirements for lane changes are roughly equivalent to those in the maintain speed scenario. 
As both need to manage vehicles in front of and behind the ego vehicle as well as to each side. 
Overall, this gives us the picture for coverage requirements for the highway driving scenario. 
We need longitudinal sensors and lateral sensors and both wide and narrow FOV sensors to do these three maneuvers, the emergency stop, maintaining speed and changing lanes. 
Already from this small set of ODD requirements we see a large variety of sensor requirements that arise. 

\subsubsection{Urban Scenario}
Let's discuss the urban scenario next. 
The urban scenario as we discussed before is a moderate volume, moderate traffic scenario with fewer lanes on the highway case but with the added complexity of pedestrians. 
There are six types of basic maneuvers here. Obviously, we can still perform emergency stop, maintain speed and lane changes but we also have scenarios such as overtaking a parked car, 
left and right turns at intersections and more complex maneuvers through intersections such as roundabouts. 
In fact, for the first three basic maneuvers, the coverage analysis is pretty much the same as the highway analysis but since we are not moving as quickly, 
we don't need the same extent for our long-range sensing. 
Let's discuss the overtake maneuver next. 
More specifically, consider a case where you have to overtake a parked car. 
Longitudinally, we definitely need to sense the parked car as well as look for oncoming traffic. 
So, we need both sensors, wide short-range sensors to detect the parked car and narrow long-range sensors to identify if oncoming traffic is approaching. 
Laterally, we'll need to observe beyond the adjacent lanes for merging vehicles as we did in the highway case. 
For intersections, we need to have near omni-directional sensing for all kinds of movements that can occur. 
Approaching vehicles, nearby pedestrians, doing turns and much more. 
Finally, for roundabouts we need a wide-range, short distance sensor laterally since the traffic is slow but 
we also need a wide-range short distance sensor longitudinally because of how movement around the roundabout occurs. 
We need to sense all of the incoming traffic flowing through the roundabout to make proper decisions. 
So we end up with this overall coverage diagram for the urban case. 
The main difference with respect to highway coverage is because of the sensing we require 
for movement at intersections and at roundabouts and for the overtaking maneuver. 
In fact, the highway case is almost entirely covered by the urban requirements. 


Let's summarize the coverage analysis. For all of the maneuvers we do, we need long range sensors which typically 
have shorter angular field of view and wide angular field of view sensors which typically have medium to short-range sensing. 
As the scenarios become more complex, we saw the need for full 360 degrees sensor coverage on the short scale out to 
about 50 meters and much longer range requirements in the longitudinal direction. 
We can also add even shorter range sensors like sonar which are useful in parking scenarios and so in the end our sensor configuration looks something like this diagram. 

To summarize, our choice of sensors should be driven by the requirements of the maneuvers we want to execute and it should include 
both long-range sensors for longitudinal dangers and wide field of view sensors for omnidirectional perception. 
The final choice of configurations also depends on our requirements for operating conditions, sensor redundancy due to failures and on budget. 
There is no single answer to which sensors are needed for a self-driving car. 

