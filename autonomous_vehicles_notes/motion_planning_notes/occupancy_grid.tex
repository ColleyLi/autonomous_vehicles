\section{Occupancy Grid Map}


This section discusses the concept of occupancy grid map. We will start by defining the occupancy grid map in detail, and understand how it can be created on an autonomous vehicle. We will then try to understand
the noise inherent in measurement data used for the creation of an
occupancy grid map. Finally, you will
see how to handle this noise in
the measurement data, by learning about
Bayesian updates of occupancy grid beliefs.

\subsection{Occupancy Grid}


An occupancy grid is a discretized grid which surrounds the current
ego vehicle position. This discretization can be done in two or three dimensions. The methods we discussed
can be applied to both two and three-dimensional problems. However, to simplify
the explanations, we will only be focusing on the 2D version. 

Each grid square of the occupancy grid indicates if a static or stationary object is present in that grid location. If so, that grid location is classified as occupied. Example of static
objects that would be classified as occupying a grid
cell can include trees, buildings, road signs,
and light poles. 

In the domain of autonomous vehicles, other objects which you might
not think of as obstacles, should also be classified
as occupying space, including all non drivable surfaces such as lawns or sidewalks. Each square of the
occupancy grid noted by $m_i$, maps to a binary value
in which the value one indicates that the square is
occupied by a static object, whilst the value zero indicates
that it is not. In this map, we can see
that the squares with trees and grass cover
are labeled as one, whereas the road is labeled zero. The resulting map
looks like this. All the occupied squares
of the grid are purple, and the rest of the map
corresponding to the drivable surfaces
is left transparent. 


Let's now discuss some of the common assumptions typically made when creating occupancy grid.
The first assumption is that  the environment that is currently measured to create this occupancy grid only
corresponds with static objects. Meaning, all dynamic objects
or moving objects must be removed from the sensor data before it is used for
occupancy grid mapping. The second assumption is that each grid cell is independent of all others. 
This assumption is made to simplify the update functions needed
to create the occupancy grid. Finally, the current
vehicle state is exactly known in relation to the occupancy map
at every time step. 

In the domain of  self-driving cars to observe distance between the car and the current state
of the world, the LIDAR sensor is
used most frequently. 

\begin{framed}
	\begin{remark}
		

As a quick reminder, the LIDAR sensor uses
pulses of light to measure the distance to
all objects surrounding the car, and returns a point cloud of measurements throughout
its field of view. 
\end{remark}
\end{framed}


In this video, we can see
the output of the LIDAR sensor. Several components of the LIDAR
data need to be filtered out before this data can be used to construct
an occupancy grid. The first step is to filter all LIDAR points which
comprise the ground plane. In this case, the ground plane is the road surface which the autonomous car
can safely drive on. Next, all points
which appear above the highest point of the vehicle
are also filtered out. This set of LIDAR points
can be ignored as they will not impede
the progression of the autonomous vehicle. Finally, all non-static
objects which had been captured by the LIDAR
need to be removed. This includes all vehicles, pedestrians, bicycles,
and animals. Once all filtering of
the LIDAR data is complete, the 3D LIDAR data will need
to be projected down to a 2D plane to be used to
construct our occupancy grid. The filtering and compression
of the LIDAR data to create accurate occupancy
grids for autonomous cars, will be covered in
a later video in this module. The LIDAR data which is now
filtered and compressed, resembles data from a high
definition 2D range sensor, which accurately
measures distance to all static objects around
the vehicle in the plane. However, there is
still a problem. After all the filtering
has been completed, there are still
major map uncertainties due to the filtering methods
used on the data, the complexity of
the data at hand, and most of all, environmental
and sensor noise. In order to handle this noise, the occupancy grid will be
modified to be probabilistic. Instead of cell i storing
a binary value for occupied, now each cell i will store a probability between
zero and one, corresponding to the certainty that the given square
is occupied. The higher the value stored, the higher the probability that the given square is occupied. To use this set of probabilities, the occupancy grid can
now be represented as a belief map denoted by
the term B-E-L are bel. To keep notations simple, Mi represents a single square
of the occupancy grid, where i can be constructed
from measurements Y, and the vehicle location X. The belief over Mi is equal to the probability
that the current cell Mi is occupied given the sensor measurements gathered
for that cell location. To convert back to a binary map, a threshold value can
be established at which a given belief is confident enough to be classified
as occupied. Any value below the set
threshold will be set to free. As an example,
the occupied square in the figure to the left has
a probability of 0.94, which classifies
the square is occupied. On the other hand, the square found on the street only has a probability of 0.12
of being occupied, and thus will be classified
as a free location. Multiple sets of
measurements can be combined from time one to time t to achieve far more accurate
belief of occupancy. In fact, we can update beliefs in a recursive manner so
that at each time step t, we use all prior information from time one onwards to
define our belief. The belief at time t
over the map cell Mi is defined as
the probability that Mi is occupied given all measurements and the vehicle
position from time one to t. To combine multiple measurements
into a single belief map, Bayes theorem can be applied. In the case of the
occupancy grid, we get a Bayesian update step that takes the following form. The distribution p
of y_t given Mi, is the probability of getting a particular measurement
given a cell Mi is occupied. This is known as
the measurement model, which you studied in
detail in course two. The belief at time T minus
one over Mi corresponds to the prior belief stored in our occupancy grid from
the previous time step. We rely on the Markov assumption, that all necessary
information for estimating cell occupancy is captured in the belief map
at each time step. So no earlier history needs to be considered in
the cell update equations. Finally, eta in this case corresponds to a normalizing constant applied
to the belief map. This is needed to scale
the results to make sure it remains
a probability distribution. Lets see an occupancy
grid in action. In this video, we will follow the autonomous vehicle as it drives out of the driveway
and down a road, while the occupancy grid
is updating in real time. The lighter grid cells
represent free squares, whereas the black grid cells
represent occupied squares. We can also see
the raw LIDAR data in red, and the filtered
outputs in orange. Notice how the map tracks
the vehicle motion which is estimated using
the same techniques as we've presented in course two. In this video, the threshold of belief needed for an object to be classified as obstructing
is set to very high, thus only large static objects are getting identified
as occupied. Lowering this threshold
value will result in more cells to be
tagged as occupied, but will lead to noisier
maps as well. All right. So let's summarize what
you've just learned. In this video, you learned the basic definition of
the Occupancy grid map, and saw how the LIDAR
sensor data can be filtered and compressed to create
an occupancy grid. You then learned how to represent the occupancy grid
as a belief map, and applied Bayesian updates to incorporate new measurements
in the occupancy grid. In the next video, we will discuss some of
the numerical problems with our belief space
map representation, and introduce the logit function as a solution to this problem. We will also look at an inverse measurement model
which is needed to create the occupancy map grid
from 2D LIDAR data using the logit function.
See you there.


\subsection{Questions}


\subsection{Assignments}