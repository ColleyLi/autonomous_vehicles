\chapter{Linear Systems}
\label{linear_systems}
Modeling of automotive systems (or any other system whatsoever) typically results in a mathematical description of it. This mathematical description is usually given in the
form of differential equations that involve the input(s) and the output(s) of the system as well as other parameters that are of interests and affect its behavior. In this chapter,
we introduce the notion of a linear system (LS) and other useful terms that will be used in subsequent chapters. We will see that a LS can be represented in different forms susch as:


\begin{itemize}

\item Differential equations
\item Transfer functions
\item State-Space models

\end{itemize} 

\section{Linear ODEs}

Let's consider the ODE:

\begin{equation}
\frac{d^{n}y}{dt^{n}} + \alpha_{1}\frac{d^{n-1}y}{dt^{n-1}} + \dots + \alpha_{n}y = \beta_{1}\frac{d^{n-1}u}{dt^{n-1}} + \dots + \beta_{n}u  
\label{linear_sys_1}
\end{equation}

where $u$ represents the input to the system and $y$ represents the output. $\alpha_i$ and $\beta_i$ are coefficients which may or may not be time dependent. In the case that the coefficients do not depend on
the time variable $t$ we have a linear time invariant (LTI) system. Otherwise the system will be time variant.


\begin{framed}

\textbf{Superposition Principle}

Let the linear equation:

\begin{equation}
L{y} = F(x)
\end{equation}

where $L$ is a linear operator. If $y_1$ and $y_2$ are solution of the linear equation, then their sum will also be a solution.

\end{framed}

Every solution to such a system can be written as the sum (because the system is linear superposition of solutions...) of a solution $y_h$to the homogeneous equation and a particular solution $y_p$:


\begin{equation}
y = y_h + y_p
\label{linear_sys_general_sol}
\end{equation}

The solution to the homogeneous solution will have the form:

\begin{equation}
y_h = \sum_{k=1}^{n} C_k e^{s_k t}  
\label{linear_sys_general_ho}
\end{equation}

where $s_k$ are the roots of the so called characteristic equation or polynomial (see section \ref{characteristic_equation}) and  $C_k$ can be determined from the initial conditions. 


Now that we have established the general form of the homogeneous
solution, let's turn our attention to the particular solution.  In order to find the particular solution $y_p$, we need the input signal $u$. Let's assume a damped sinusoidal input signal of the following form

\begin{equation}
u(t) = e^{st}  
\label{linear_sys_input}
\end{equation}

where $s$ is a complex variable.  The particular solution is assumed to have the follwoing general form:


\begin{equation}
y_p= G(s)e^{st}  
\label{linear_sys_general_pa}
\end{equation}

The function $G(s)$ is called the transfer function TF. Thus, the general solution of the equation \ref{linear_sys_1} is


\begin{equation}
y (t) = \sum_{k=1}^{n} C_k e^{s_k t} + G(s)e^{st}  
\label{linear_sys_general_sol_II}
\end{equation}

The first part gives the dependency due to the initial conditions. The second part is due to the input signal.


\subsection{Characteristic Equation}
\label{characteristic_equation}

In equation \ref{linear_sys_general_sol_II} the $s_k$s are the roots of the characteristic equation or polynomial $\alpha(s)$. This polynomial is formed from the coefficients of the system
\ref{linear_sys_1}. Hence, for the system

\begin{equation}
\frac{d^{n}y}{dt^{n}} + \alpha_{1}\frac{d^{n-1}y}{dt^{n-1}} + \dots + \alpha_{n}y = \beta_{1}\frac{d^{n-1}u}{dt^{n-1}} + \dots + \beta_{n}u  
\nonumber
\end{equation} 

we will have

\begin{equation}
\alpha(s) = s^n +\alpha_{1}s^{n-1} + \dots + \alpha_n   
\label{linear_sys_charact_pol}
\end{equation}
Thus, the $s_k$ will be the roots of the equation


\begin{equation}
\alpha(s) = 0 
\label{linear_sys_charact_pol_eq}
\end{equation}

Concretely, the solutions $s_k$ are called the \textbf{poles}  of the system and play a crucial role in the stability of the solution:

\begin{itemize}
\item If all the roots of $\alpha(s)=0$ have $Re(s_k) < 0 $ then the system is asymptotically stable
\item If any of the roots of $\alpha(s)=0$ has $Re(s_k) > 0 $ then the system is  unstable
\end{itemize}

\begin{framed}
\label{linear_sys_exe_1}

\textbf{Example}

Consider the system

\begin{equation}
\dot{x} = x +u \nonumber
\end{equation}

Form the characteristic equation. Is the solution stable or not?

\textbf{Answer}

The characteristic equation of the system above is

\begin{equation}
\alpha(s) = s -1 =0 \nonumber
\end{equation}

The only root to this equation is $s=1$ and since $Re(s) > 0$ the system is not asymptotically stable.

\end{framed}



\subsection{Laplace Transformation}


\section{State-Space models}

A state-space formulation of a linear system has the following form



\begin{eqnarray}
\dot{x} = Ax + Bu \\
y = Cx + Du
\label{linear_space_form}
\end{eqnarray}

where

\begin{itemize}
\item $x \in R^{n}$ is the state vector
\item $y \in R^{n}$ is the output vector
\item $u \in R^{p}$ is the input vector 
\item $A \in R^{n \times n}$ is the matrix describing the dynamics 
\item $B \in R^{n \times p}$ is the matrix describing the input 
\item $C \in R^{q \times n}$ is the output or sensor matrix 
\item $D \in R^{q \times p}$ is the direct matrix 
\end{itemize}

Just like in equation \ref{linear_sys_1}, if the coefficient matrices do not depend on time, we have an LTI system.



\section{Questions}


\textbf{Question 1}

Consider the following model:

\begin{equation}
\dot{x} = x +u \nonumber
\end{equation}

We saw in example \ref{linear_sys_exe_1} that the solution to this system is asymptotically stable. Cast the system in the

\begin{equation}
\frac{dx}{dt} = Ax + Bu \nonumber
\end{equation}

form and argue again about its stability.


\textbf{Answer}

We can write the given system in the form above if we assume that

\begin{equation}
A = [1] ~~ \text{and} ~~ B = [1]  \nonumber
\end{equation}

The stability depends on the eignevalues of $A$ and  here we have only one eigenvalue which is $\lambda = 1$. Thus, since $Re(\lambda) > 0 $ the system is not asymptotically stable as we concleded in example 
\ref{linear_sys_exe_1}


\textbf{Question 2}

Consider the following model:

\begin{equation}
\frac{d^2x}{dt^2} + 2\frac{dx}{dt} + x  = u \nonumber
\end{equation}

compute the poles of the equation and argue about the stability.


\textbf{Answer}

The poles of the equation are the solutions of the characteristic equation:

\begin{equation}
\alpha(s) = s^2 + 2s +1 =0  \nonumber
\end{equation}

The discriminant of the quadratic equation is

\begin{equation}
\Delta = b^2 - 4ac =0  \nonumber
\end{equation}

Hence, the characteristic polynomial has only one real solution given by

\begin{equation}
s = \frac{-b}{2a} = -1  \nonumber
\end{equation}

Thus, since $Re(s) < 0 $ the system is asymptotically stable.



\textbf{Question 3}

Determine the poles of the following equations:


\begin{eqnarray}
\text{A)} ~~ \frac{d^2x}{dt^2} + 2\frac{dx}{dt} + x = u \nonumber \\
\text{B)} ~~ \frac{d^2x}{dt^2} + 2\frac{dx}{dt} + 2x = u \nonumber \\
\text{C)} ~~ \frac{d^2x}{dt^2} + 3\frac{dx}{dt} + 2 x  = u \nonumber \\
\end{eqnarray}


For the the model A we have:

\begin{equation}
\alpha(s) = s^2 + 2s +1 =0  \nonumber
\end{equation}

Thus, only one pole $s=-1$. For model B,

\begin{equation}
\alpha(s) = s^2 + 2s +2 =0  \nonumber
\end{equation}

The discriminant $\Delta = -4 <0$.  Thus, the system has two complex solution

\begin{eqnarray}
s_1 = -1 +i  \nonumber \\
s_2 = -1 -i  \nonumber \\
\end{eqnarray}

For model C, 

\begin{equation}
\alpha(s) = s^2 + 3s +2 =0  \nonumber
\end{equation}

and the solutions are

\begin{eqnarray}
s_1 = -1   \nonumber \\
s_2 = -2  \nonumber \\
\end{eqnarray}








