\chapter{Getting Started}
\label{getting_started}
Welocome! This is a book about applied Computational Statistics. Before we dive into the theory and start coding this
theory, it could to grasp the very basics of what we are trying to. This chapter introduces various notions and will 
help you set up your environment for the computational assignements.

\section{What is Statistics}
The word statistics has two meanings \cite{Mann2010}. In the more common usage, statistics refers to numerical facts. The numbers that represent the income of a family, the age of a student, the percentage of passes completed by the quarterback of a football team, and the starting salary of a typical college graduate are examples of statistics in this sense of the word. The second meaning of statistics refers to the field or discipline of study. In this sense of
the word, statistics is defined as follows \cite{Mann2010}.

\subsubsection{Definition}
\label{stats_definition}
Statistics is a group of methods used to collect, analyze, present and interpret data and to make decisions.  

Like almost all fields of study, statistics has two aspects: theoretical and applied. Theoretical or mathematical statistics deals with the development, derivation, and proof of statistical
theorems, formulas, rules, and laws. Applied statistics involves the applications of those theorems, formulas, rules, and laws to solve real-world problems. 

This text is concerned with applied statistics and not with theoretical statistics. By the time you finish studying this book, you
will have learned how to think statistically and how to make educated guesses.

\section{Types of Statistics}
\label{types_stats}
Broadly speaking, the field of applied statistics can be divided into two groups \cite{Mann2010}. Namely

\begin{itemize}
\item Descriptive Statistics
\item Inferential Statistics
\end{itemize}

Descriptive Statistics consists of methods for organizing, displaying and describing data using tables, graphs and summary measures. On the other hand, Inferential Statistics consists of methods that use sample results to help make decisions or predictions about a population.

Interrelated  with the scientific field of Statistics is the field of Probability. In general, the probability, which gives a measurement of the likelihood that an event will occur, acts as a link between Descriptive Statistics and Inferential Statistics.
Probability is used to make statements about the occurrence or non-occurrence of an event under uncertain conditions.

\section{Types Of Variables}
\label{types_variables}
Statistics deals with the understanding of the influence of variables (also called features) over the population under investigation. Thus the term variable plays a crucial role  for us. The incomes of families, heights of persons, gross sales
of companies, prices of college textbooks, makes of cars owned by families, number of accidents, and status (freshman, sophomore, junior, or senior) of students enrolled at a university are examples of variables. 

A variable may be classified as qualitative of quantitative \cite{Mann2010}, \cite{Tsimbos1999}. The defining feature between these two categories of variables is that latter can be measured numerically whilst the former cannot. Figure \ref{types_of_variables} shows the types of variables we will be dealing with and their sub-categories.

\begin{figure}[!htb]
\begin{center}
\includegraphics[scale=0.280]{img/intro/types_of_variables.jpg}
\end{center}
\caption{Types of variables used in Statistics.}
\label{types_of_variables}
\end{figure}


 Let's elaborate a bit more on these two definitions as it is very important to understand them.

\subsection{Quantitative Variables}
\label{quantitative_variables}
We have already mentioned what a quantitative variable is. Some examples of this type of variables are the weight or the height of the individuals of a population or the number of children or number of cars an individual of a population may have. All these variables can be expressed numerically however observer that variables like number of children or number of cars, we can only give answers of the form 0 or 1 or 2 etc. In contrast, variables like weight or height can be expressed with as much accuracy as we desire. Thus, we arrive at a new separation point. Namely, quantitative (also known as numerical) variables are categorized into discrete and continuous variables

\subsection{Qualitative Variables}
\label{qualitative_variables}

\section{Types of Variables in Python} 
\label{types_variables_python}

\section{Types of Variables in R}
\label{types_variables_r}


\section{Measurements}
\label{measurements}
In this section, we introduce some more terms.

\begin{itemize}

\item Observations: typically correspond to the measurements of a variable
\item Unknowns: These are the parameters we are interested in
\item Estimating: This will be the link 
\end{itemize}

\section{Summary}

\section{Exercises} 



